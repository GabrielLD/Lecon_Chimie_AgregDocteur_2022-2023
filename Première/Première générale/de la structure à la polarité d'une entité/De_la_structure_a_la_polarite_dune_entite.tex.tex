% IMPORTANT: PLEASE USE XeLaTeX FOR TYPESETTING
\documentclass{beamer}
\input{pckg_beamer.tex}
\usepackage{tikz}
\usetikzlibrary{matrix}
% document body
\begin{document}
\maketitle
% --------- Sommaire ---------
% \begin{frame}
%     \tableofcontents
% \end{frame}      
% ----------------------------

\begin{frame}{Bulletin officiel}
    \begin{figure}
        \centering
        \includegraphics[width=.7\textwidth]{Bo.png}
    \end{figure}
\end{frame}

\begin{frame}
    \tableofcontents
\end{frame}

\begin{frame}{Biblio}
    \begin{itemize}
        \item nathan 1ere 2019
        \item Belin 
        \item Hatier 2011 1ere 
    \end{itemize}
\end{frame}
\section{Schéma de Lewis}
\subsection{Doublets de Valence}

\begin{frame}
    \begin{figure}
        \centering
        \includegraphics[width=.5\textwidth]{BO.png}
    \end{figure}
\end{frame}
\subsection{Construction d'un schéma de Lewis}
\subsection{Géométrie}

\begin{frame}
    \begin{itemize}
        \item     \url{https://www.ccdc.cam.ac.uk/}
        \item \url{http://www.crystallography.net/cod/index.php}
    \end{itemize}

\end{frame}
\section{Polarité d'une molécule}
\subsection{Notion d'électronégativité}
\subsection{Molécule polaire ou apolaire}
\end{document}